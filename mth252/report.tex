
\documentclass[10pt,letterpaper,cm]{hmcpset}
\usepackage[top=0.5in,bottom=1in,right=1in,left=0.75in]{geometry}
\usepackage{polynom,multicol,graphicx, enumerate, cancel, amsmath, amsthm}
\name{Jonathon Sonesen}
\class{MTH 252}
\duedate{\today}
\assignment{Group Report}
\extraline{Collaborators: Nicole Horsley}

\setlength{\columnsep}{0.5cm}
\setlength{\columnseprule}{00pt}
\begin{document}
\section*{Descripton:}
In appendix G the method for using partial fractions as a way to integrate a function 
that exists as a ratio of polynomials is explained. This is achieved by breaking the ratio down into 
a sum of simpler fractions. This operation can be performed as long as the degree of the polynomial in the
denominator is greater the polynomial in the numerator, this is because the ratio must be a proper fraction.
However, if the opposite is true and the fraction is improper, using long division with polynomials can create
a proper fraction. In general there are four cases for using partial fractions to integrate a function.
These cases are as follows:\\
\section*{Cases:}
\\
\begin{problem}
  \begin{aligned*}
    & \text{Case I: The denominator is a product of distinct linear factors.}
    \begin{eqnarray*}
      \frac{P(x)}{(a_1x+b_1)(a_2x+b_2)(a_kx+b_k)} =   \frac{A_1}{(a_1x+b_1)} +   \frac{A_2}{(a_2x+b_2)}
      +\cdot\cdot\cdot+   \frac{A_k}{(a_kx+b_k)}
    \end{eqnarray*}
    \\
    \\
    & \text{Case II: The denominator is a product of linear factors, some of which are repeated.}
    &\begin{align*}
      \frac{P(x)}{(ax+b)^r} =  \frac{A_1}{a_1 \cdot x+b_1} + \frac{A_2}{(a_1\cdot x+b_2)^2} +\cdot\cdot\cdot+ \frac{A_r}{(a_1\cdot x+b_1)^r}
    \end{align*}
    \\
    \\ 
    \text{Case III: The denominator contains irreducible quadratics none of which are repeated.}
    \begin{eqnarray*}
      \frac{P(x)}{ax^2+bx + c} = \frac{Ax+B}{ax^2+bx + c}
    \end{eqnarray*}
    \\
    \\
    & \text{Case IV: The denominator contains a repeated irreducible quadratic factor.}
    \begin{eqnarray*}
     \frac{P(x)}{(ax^2+bx + c)^r} = \frac{A_1x +B_1}{ax^2+bx + c} + 
     \frac{A_2x+B_2}{(ax^2+bx + c)^2} + \cdot\cdot\cdot + \frac{A_rx+B_r}{(ax^2+bx + c)^r}
    \end{eqnarray*}
  \end{aligned*}
\end{problem}
\\
\\
\newpage
\section*{Example:}


\begin{problem}
To exemplify the process of evaluating integrals using partial fractions consider:

  \begin{equation*}
    \int\frac{1}{(x+a)(x+b)}dx 
  \end{equation*}
\end{problem}\\
  This function falls under Case I:
  \begin{align*}
      \frac{1}{(x+a)(x+b)} =& \frac{A}{x+a} + \frac{B}{x+b} & \text{by case 1 for partial fractions}\\
       =&A(x+b) + B(x+a) &\text{by multiplying both sides by the denominator}\\
   A+B =& 0 \\
   bA + aB=& 1 \\
   A =&\frac{1}{b-a} \\ 
   B =&\frac{1}{b-a} \\
  \end{align*}
\\
This gives us,
\begin{align*}
  \int\frac{1}{(x+a)(x+b)}dx=&\frac{1}{b-a}\left(  \int\left[ \frac{1}{x+a}- \frac{1}{x+b}\right] 
  dx\right)\\
  =&\frac{1}{b-a}\left(\ln|x+a|-\ln|x+b|\right)+C\\
  =& \frac{1}{b-a}\left(\ln\frac{|x+a|}{|x+b|}\right)+C \\
\end{align*}
\newpage
\section*{Problems:}\\
\\
\begin{problem}[1]
  Evaluate the integral:
    \begin{equation*}
      \int\frac{x-1}{x^2 + 3x + 2}dx
    \end{equation*}
\end{problem}\\

\begin{equation*}
  \int_{0}^{1}\frac{x-1}{x^2+3x+2}dx = \int_{0}^{1}\frac{x-1}{(x+1)(x+2)}dx
\end{equation*}
Consider:
\begin{center}
  \begin{aligned}
      \frac{x-1}{(x+1)(x+2)} & = \frac{A}{(x+2)} + \frac{B}{(x+1)} \\
      & = A(x+1) + B(x+2) \\
  \end{aligned}\\
\end{center}
Let x:= -1
\begin{align*}
  -2&=A(0)+B(1)\\
  B&= -2
\end{align*}
Let x:= -2
\begin{align*}
  -3&= A(-1) + B(0)\\
  A&= 3
\end{align*}
Since, A&=& 3~and~B&=& -2
\begin{align*}
  \int_{0}^{1}\frac{x-1}{(x+1)(x+2)}dx &=\int_{0}^{1}\frac{-2}{x+1} + \frac{3}{x+2}  dx\\
  \\
  &= [~-2\ln{|x+1|} + 3\ln{|x+2|}~]_{0}^{1}\\
  &=-2\ln{(1+1)} + 3\ln{(1+2)}-(-2\ln{(0+1)} - 3\ln{(0+2)})\\
  &=3\ln{(3)}-5\ln{(2)}\\
  &=\ln{\frac{27}{32}}\\  
\end{align*}
\newpage
\begin{problem}[2]
  Evaluate the integral:
  \begin{equation*}
    \int\frac{x^2 -5x + 16}{(2x+1)(x-2)^2}dx 
  \end{equation*}
\end{problem}\\
\\
Consider,
\begin{align*}
  \frac{x^2 -5x + 16}{(2x+1)(x-2)^2} &
  =\frac{A}{2x+1}+\frac{B}{x-2}+\frac{C}{(x-2)^2} \\ \\
  x^2 -5x + 16 &= A(x-2)^2 + B(x-2)(2x+1)+C(2x+1)
\end{align*}
Let x:=2
\begin{align*}
  (2)^2 -5(2) + 16 &= A(2-2)^2 + B(2-2)(2(2)+1)+C(2(2)+1)\\
  10 &= 5C\\
  C  &=2
\end{align*}
Now,
\begin{align*}
  x^2 -5x + 16 &= Ax^2-4Ax+4A + 2Bx^2 -3Bx-2B+2Cx + C\\
  &= x^2(A+2B) + x(-4A - 3B + 2C)+(4A-2B+C)\\
\end{align*}
Using the terms and their corresponding coefficients,
\begin{eqnarray}
  &A+~2B=1 &\\
  &-4A-~3B=-9&\\
  & 4A-~2B=14&
\end{eqnarray}
Now add (2) and (3) vertically,
\begin{eqnarray*}
  &-4A-~3B  &= -9\\
  & 4A-~2B  &= 14\\
%+____
 + \cline{2-3}
  & 0A  -5B  &=  5\\
  & 0A  + B  &= -1
\end{eqnarray*}
Solve for A,
\begin{equation*}
  A+2(-1)=1\\
\end{equation*}
  In summary,\begin{eqnarray*}
    A=&3\\
    B=&-1\\
    C=&2\\
  \end{eqnarray*}
Since,
\begin{align*}
  \frac{x^2 -5x + 16}{(2x+1)(x-2)^2} &
  =\frac{3}{2x+1}-\frac{1}{x-2}+\frac{2}{(x-2)^2} \\ \\
\end{align*}
We have,
  \begin{align*}
    \int\frac{x^2 -5x + 16}{(2x+1)(x-2)^2}dx
    =&\int\frac{3}{2x+1}-\frac{1}{x-2}+\frac{2}{(x-2)^2}dx~&~u=x-2~&w=2x+1\\
    =&\frac{3}{2}\int\frac{1}{w}dw - \int\frac{1}{u}du + \int\frac{2}{u^2}du~&~du=dx~&dw=2dx
  \\=&\frac{3}{2}\ln(|w|)-ln(|u|)-\frac{2}{u} + C 
  \\=& \frac{3}{2}\ln(|2x+1|)-ln(|x-2|)-\frac{2}{x-2} + C 
  \end{align*}
\newpage
\begin{problem}[3]
  Evaluate the integral:
  \begin{equation*}
    \int\frac{x^2 - x + 6}{x^3+3x}dx 
  \end{equation*}
\end{problem}\\
Consider,\\
\begin{align*}
  \frac{x^2 - x + 6}{x^3+3x} &= \frac{A}{x} + \frac{Bx+C}{x^2+3}\\
  \\x^2-x+6=&A(x^2+3)+(Bx+C)(x)\\
           =&Ax^2+3A+Bx^2+Cx\\
           =&x^2(A+B)+x(C) + 3A
\end{align*}
By the terms corresponding coefficients,
\begin{align*}
          A+B &= 1 \rightarrow& B&=-1\\
          3A  &=6 \rightarrow& A&=2\\  
           && C &=1\\
\end{align*}
Since,
\begin{align*}
  \frac{x^2 - x + 6}{x^3+3x} &= \frac{2}{x} - \frac{x+1}{x^2+3}\\
\end{align*}
The integral is,
\begin{align*}
  \int\frac{2}{x}dx - \int\frac{x+1}{X^2+3}~dx =&~2\ln(|x) - \int\frac{x}{x^2+3}dx
  - \int\frac{1}{x^2+3}dx &u=&~x^2+3 \\
  =&~2\ln\left( |x| \right)-\frac{1}{2}\int\frac{1}{u}du
  - \int\frac{1}{x^2+3}dx &du=&~2xdx \\
  =&~2\ln\left( |x| \right)-\frac{1}{2} \ln(|x^2+3|)
  - \frac{1}{\sqrt{3}}\int\frac{1}{\left( \frac{x}{\sqrt{3}} \right)^2+1} dx \\
  =&~2\ln\left( |x| \right)-\frac{1}{2} \ln(|x^2+3|)
  - \frac{1}{\sqrt{3}}tan^{-1}\left( \frac{x}{\sqrt{3}}\right) + C \\
\end{align*}

\newpage
\begin{problem}[4]
  Using long division first evaluate the integral:
  \begin{equation*}
    \int\frac{x^3 + 1}{x^2-4}dx 
  \end{equation*}
\end{problem}\\
\\
Consider the division,
\begin{equation*}
  \polylongdiv{x^3+1}{x^2-4}
\end{equation*}
This gives us,
\begin{align*}
  \int\frac{x^3 + 1}{x^2-4}dx
  =& \int xdx + \int\frac{4x+1}{x^2-4}dx  \\
  =& \frac{1}{2}x^2 + \int\frac{4x+1}{(x+2)(x-2)}dx  \\
\end{align*}
By case I:
\begin{align*}
  \frac{4x+1}{(x+2)(x-2)}
  =&\frac{A}{x-2}+\frac{B}{x+2}\\ 
  =&A(x+2)+B(x-2)
\end{align*}
Let x:=$2$
\begin{align*}
  4(2)+1=&A(2+2)+B(0) \\
  A     =&\frac{9}{4}
\end{align*}
Let x:= $-2$
\begin{align*}
  4(-2)+1=&A(-2+2)+B(-2-2) \\
  B     =&\frac{7}{4}
\end{align*}
Thus,

\begin{align*}
  \frac{1}{2}x^2 + \int\frac{4x+1}{(x+2)(x-2)}dx
  =& \frac{1}{2}x^2 + \int\frac{\frac{7}{4}}{(x+2)}dx + \int\frac{\frac{9}{4}}{(x-2)}dx \\
  =& \frac{1}{2}x^2 + \frac{7}{4}\ln(|x+2|) + \frac{9}{4}\ln(|x-2|) + C \\
\end{align*}
Sources: Wolfram Alpha, TI-89 Titanium, James Stewart Calculus Concepts 4th Edition
\end{document}
