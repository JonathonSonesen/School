
\documentclass[10pt,letterpaper, cm]{hmcpset}
\usepackage[top=0.5in,bottom=1in,right=1in,left=0.75in]{geometry}
\usepackage{multicol,graphicx, enumerate, cancel, amsmath, amsthm}
\usepackage{tikz}
\name{Jonathon Sonesen}
\class{MTH 252}
\duedate{\today}
\assignment{Group Report}
\extraline{Collaborators: Nicole Horsley}

\setlength{\columnsep}{0.5cm}
\setlength{\columnseprule}{00pt}
\begin{document}
In appendix G the method for using partial fractiions as a way to integrate a function 
that exists as a ratio of polynomials. This is achieved by breaking the ratio down into 
a sum of simpler fractions. This operation can be performed as long as the degree of the polynimial in the
denominator is greater the polynomial in the numerator, this is becuase the ratio must be a proper fraction.
However, if the opposite is true and the fraction is improper using long division with polynomials can create
a proper fraction. In general there are four cases for using partial fractions to integrate a fuction.
These cases are as follows:\\
\\
\begin{problem}
  \begin{aligned*}
    & \text{Case I: The denominator is a product if distinct linear factors.}
    \begin{eqnarray*}
      \frac{P(x)}{(a_1x+b_1)(a_2x+b_2)(a_kx+b_k)} =   \frac{A_1}{(a_1x+b_1)} +   \frac{A_2}{(a_2x+b_2)}
      +\cdot\cdot\cdot+   \frac{A_k}{(a_kx+b_k)}
    \end{eqnarray*}
    \\
    \\
    & \text{Case II: The denominator is a product of linear factors, some of which are repeated.}
    &\begin{eqnarray*}
      \frac{P(x)}{(ax+b)^r} =  \frac{A_1}{ax+b} + \frac{A_2}{ax+b} +\cdot\cdot\cdot+ \frac{A_r}{ax+b}
    \end{eqnarray*}
    \\
    \\ 
    \text{Case III: The denominator contains irreduciible quadratics none of which are repeated.}
    \begin{eqnarray*}
      \frac{P(x)}{ax^2+bx + c} = \frac{Ax+B}{ax^2+bx + c}
    \end{eqnarray*}
    \\
    \\
    & \text{Case IV: The denominator contains a repeated irreducible qudratic factor.}
    \begin{eqnarray*}
     \frac{P(x)}{(ax^2+bx + c)^r} = \frac{A_1x +B_1}{ax^2+bx + c} + 
     \frac{A_2x+B_2}{(ax^2+bx + c)^2} + \cdot\cdot\cdot + \frac{A_rx+B_r}{(ax^2+bx + c)^r}
    \end{eqnarray*}
  \end{aligned*}
\end{problem}
\\
\\
To exemplify the process of evaluating integrals using impartial fractions consider:

  \begin{equation*}
    \int\frac{1}{(x+a)(x+b)}dx 
  \end{equation*}

\begin{problem}


\end{problem}
\end{document}
