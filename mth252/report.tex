
\documentclass[10pt,letterpaper,cm, fleqn]{hmcpset}
\usepackage[top=0.5in,bottom=1in,right=1in,left=0.75in]{geometry}
\usepackage{multicol,graphicx, enumerate, cancel, amsmath, amsthm}
\usepackage{xlop}
\name{Jonathon Sonesen}
\class{MTH 252}
\duedate{\today}
\assignment{Group Report}
\extraline{Collaborators: Nicole Horsley}

\setlength{\columnsep}{0.5cm}
\setlength{\columnseprule}{00pt}
\begin{document}
\section*{Descripton:}
In appendix G the method for using partial fractiions as a way to integrate a function 
that exists as a ratio of polynomials is explained. This is achieved by breaking the ratio down into 
a sum of simpler fractions. This operation can be performed as long as the degree of the polyniomial in the
denominator is greater the polynomial in the numerator, this is becuase the ratio must be a proper fraction.
However, if the opposite is true and the fraction is improper using long division with polynomials can create
a proper fraction. In general there are four cases for using partial fractions to integrate a fuction.
These cases are as follows:\\
\section*{Cases:}
\\
\begin{problem}
  \begin{aligned*}
    & \text{Case I: The denominator is a product if distinct linear factors.}
    \begin{eqnarray*}
      \frac{P(x)}{(a_1x+b_1)(a_2x+b_2)(a_kx+b_k)} =   \frac{A_1}{(a_1x+b_1)} +   \frac{A_2}{(a_2x+b_2)}
      +\cdot\cdot\cdot+   \frac{A_k}{(a_kx+b_k)}
    \end{eqnarray*}
    \\
    \\
    & \text{Case II: The denominator is a product of linear factors, some of which are repeated.}
    &\begin{eqnarray*}
      \frac{P(x)}{(ax+b)^r} =  \frac{A_1}{ax+b} + \frac{A_2}{ax+b} +\cdot\cdot\cdot+ \frac{A_r}{ax+b}
    \end{eqnarray*}
    \\
    \\ 
    \text{Case III: The denominator contains irreduciible quadratics none of which are repeated.}
    \begin{eqnarray*}
      \frac{P(x)}{ax^2+bx + c} = \frac{Ax+B}{ax^2+bx + c}
    \end{eqnarray*}
    \\
    \\
    & \text{Case IV: The denominator contains a repeated irreducible qudratic factor.}
    \begin{eqnarray*}
     \frac{P(x)}{(ax^2+bx + c)^r} = \frac{A_1x +B_1}{ax^2+bx + c} + 
     \frac{A_2x+B_2}{(ax^2+bx + c)^2} + \cdot\cdot\cdot + \frac{A_rx+B_r}{(ax^2+bx + c)^r}
    \end{eqnarray*}
  \end{aligned*}
\end{problem}
\\
\\
\newpage
\section*{Example:}
To exemplify the process of evaluating integrals using partial fractions consider:


\begin{problem}

  \begin{equation*}
    \int\frac{1}{(x+a)(x+b)}dx 
  \end{equation*}
  This function falls under Case I:
  \begin{align*}
      \frac{1}{(x+a)(x+b)} &= \frac{A}{x+a} + \frac{B}{x+b} & \text{by case 1 for partial fractions}\\
     & =A(x+b) + B(x+a) &\text{by multiplying both sides by the denominator}\\
     & = 
  \end{align*}
\end{problem}\\
\\
\section*{Problems:}\\
\\
\begin{problem}[1]
  Evaluate the integral:
    \begin{equation*}
      \int\frac{x-1}{x^2 + 3x + 2}dx
    \end{equation*}
\end{problem}\\

\begin{equation*}
  \int_{0}^{1}\frac{x-1}{x^2+3x+2}dx = \int_{0}^{1}\frac{x-1}{(x+1)(x+2)}dx
\end{equation*}
Consider:
\begin{center}
  \begin{aligned}
      \frac{x-1}{(x+1)(x+2)} & = \frac{A}{(x+2)} + \frac{B}{(x+1)} \\
      & = A(x+1) + B(x+2) \\
  \end{aligned}\\
\end{center}
Let x:= -1
\begin{align*}
  -2&=A(0)+B(1)\\
  B&= -2
\end{align*}
Let x:= -2
\begin{align*}
  -3&= A(-1) + B(0)\\
  A&= 3
\end{align*}
Since, A&=& 3~and~B&=& -2
\begin{align*}
  \int_{0}^{1}\frac{x-1}{(x+1)(x+2)}dx &=\int_{0}^{1}\frac{-2}{x+1} + \frac{3}{x+2}  dx\\
  \\
  &= [~-2\ln{|x+1|} + 3\ln{|x+2|}~]_{0}^{1}\\
  &=-2\ln{(1+1)} + 3\ln{(1+2)}-(-2\ln{(0+1)} - 3\ln{(0+2)})\\
  &=3\ln{(3)}-5\ln{(2)}\\
  &=\ln{\frac{27}{32}}\\  
\end{align*}
\newpage
\begin{problem}[2]
  Evaluate the integral:
  \begin{equation*}
    \int\frac{x^2 -5x + 16}{(2x+1)(x-2)^2}dx 
  \end{equation*}
\end{problem}\\
\\
Consider,
\begin{align*}
  \frac{x^2 -5x + 16}{(2x+1)(x-2)^2} &
  =\frac{A}{2x+1}+\frac{B}{x-2}+\frac{C}{(x-2)^2} \\ \\
  x^2 -5x + 16 &= A(x-2)^2 + B(x-2)(2x+1)+C(2x+1)
\end{align*}
Let x:=2
\begin{align*}
  (2)^2 -5(2) + 16 &= A(2-2)^2 + B(2-2)(2(2)+1)+C(2(2)+1)\\
  10 &= 5C\\
  C  &=2
\end{align*}
Now,
\begin{align*}
  x^2 -5x + 16 &= Ax^2-4Ax+4A + 2Bx^2 -3Bx-2B+2Cx + C\\
  &= x^2(A+2B) + x(-4A - 3B + 2C)+(4A-2B+C)\\
\end{align*}
Using the corresponding terms and their coefficients,
\begin{eqnarray*}
  &A+~2B=1\\
  &-4A-~3B=-9\\
  & 4A-~2B=14\\
\end{eqnarray*}
\opmult{384}{56}\qquad

\begin{problem}[3]
  Evaluate the integral:
  \begin{equation*}
    \int\frac{x^2 - x + 6}{x^3+3x}dx 
  \end{equation*}
\end{problem}\\

\begin{problem}[4]
  Using long division first evaluate the integral:
  \begin{equation*}
    \int\frac{x^3 + 1}{x^2-4}dx 
  \end{equation*}
\end{problem}\\
\\
Sources: Wolfram Alpha, TI-89 Titanium
\end{document}
