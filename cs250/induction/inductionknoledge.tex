\documentclass{article}
\usepackage[utf8]{inputenc}

\title{CS250 Induction Knowledge Test}
\author{jonathon.sonesen }
\date{March 2015}

\begin{document}

\maketitle
1
\section{\exists n \in \mathbb{Z},~ n \geq 12 ~\rightarrow~ \exists a,b \in \mathbb{Z} ~\mid~ a \geq 0 \land b \geq 0 \land n ~\neq~ 7a ~+~ 3b}

\begin{array}{l l l}
Row & Statement & Comment \\
1 & let n:=12 & \text{counter example} \\
2 & a:= 1 & \text{} \\
3 & b:= 1 & \text{} \\
4 & 12 = 7 \cdot 1 + 3 \cdot 1 & \text{by~substitution} \\
5 & 7 + 3 = 10& \text{by~addition} \\
6 & 12 \neq 10& \text{by~definition} \\
7 & \therefore \exists n \in \mathbb{Z},~ n \geq 12 ~\rightarrow~ \exists a,b \in \mathbb{Z} ~\mid~ a \geq 0 \land b \geq 0 \land n ~\neq~ 7a ~+~ 3b& \text{by~definition} \\

\end{array}

2
\section{\forall n \in \mathbb{Z}, ~n \geq 1~\rightarrow~\sum \limits_{i=1}^{n} \left ( 5i ~-~ 4	\right )~=~\frac{n(5n ~-~ 3)}{2}}

\begin{array}{l l l}
Row & Statement & Comment \\
1 & let~n:=1 & \text{Base~case} \\
2 & 5 \cdot 1 ~-~ 4 ~=~ 1& \text{Left~side} \\
3 &  \frac{1(5 \cdot 1 ~-~ 3)}{2} & \text{Right side} \\
4 & \frac{2}{2} ~=~ 1 & \text{math} \\
5 &	\sum \limits_{i=1}^{n~+~1} \left (5i ~-~ 4\right)= \sum \limits_{i=1}^{n} (5i ~-~ 4)+ (5(n~+~1)~-~4)& \text{Decomposition} \\
6 & \sum \limits_{i=1}^{n} \left (5i ~-~ 4\right)= \frac{(n~+~1)(5(n~+~1)- 3)}{2}& \text{Substitution step} \\
7 & \sum \limits_{i=1}^{n~+~1} \left (5i ~-~ 4\right)= \sum \limits_{i=1}^{n} (5i ~-~ 4)+ (5(n~+~1)~-~4)~=~\frac{n(5n-3)}{2}+ (5(n~+~1)~-~4)& \text{by~hypothesis} \\
8 & \frac{n(5n~-~3)}{2} + 5n+1& \text{}\\\\
9 & \frac{n(5n~-~3)~+~2(5k+1)}{2}& \text{LCM}\\\\
10 & \frac{5n^2~-~3n~+~10n~+~2}{2} & \\\\
11 &\frac{5n^2~+~7n~+~2}{2} &\\\\
12 & \frac{5n^2~+~5n~+~2n~+~2}{2}&\\\\
13 & \frac{5n(n+1)~+~2(n+1)}{2}&\\\\
14 & \frac{(n~+~1)5(n~+~2)}{2}& \text{algebra}\\\\
15 & \frac{(n~+~1)(5(n~+~1)-3)}{2} ~=~\frac{n(5n ~-~ 3)}{2} & \\\\
16 &\therefore \forall n \in \mathbb{Z}, ~n \geq 1	~\rightarrow~\sum \limits_{i=1}^{n} \left (5i ~-~ 4\right )~=~\frac{n(5n ~-~ 3)}{2}&
\end{array}

3
\section{\exists n \in \mathbb{Z}, ~n \geq 2 ~\land~\prod \limits_{i=1}^{n} \left (1 - \frac{1}{i^2}\right )\neq \frac{n+1}{2n}}

\begin{array}{l l l}
Row & Statement & Comment \\
1 & let~n:=2 & \text{base case} \\
2 & 1 - \frac{1}{2^2}= \frac{3}{4}& \text{left~side} \\
3 & \frac{2~+1}{2\cdot2}=\frac{3}{4} & \text{right~side} \\

4

&   
	\prod \limits_{i=1}^{n} \left (1 - \frac{1}{i^2}\right )~=~\frac{n+1}{2n} ~\rightarrow~\prod \limits_{i=1}^{n+1} \left (1 - \frac{1}{i^2}\right ) = \frac{(n+1) + 1}{2(n+1)}
&	\text{} \\

5 

&
	\prod \limits_{i=1}^{n+1} \left (	1 - \frac{1}{i^2}\right )~=~\prod \limits_{i=1}^{n} \left (1 - \frac{1}{i^2}}\right )(1-\frac{1}{(n+1)^2})& \text{decompose}\\
6

&
    \prod \limits_{i=1}^{n} \left (1 - \frac{1}{i^2}\right )~=~	(\frac{n+1}{2n})(1-\frac{1}{(n+1)^2})
&
\text{substitution} \\

7   
& 
    (\frac{n+1}{2n})(\frac{(n+1)^2 - 1}{(n+1)^2})
& 
\text{} \\

8
&
    \frac{(n+1)(n^2 + 2n + 1 - 1)}{(2n)(n+1)^2}
&\\

9
&
    \frac{n^2 + 2n}{(2n)(n+1)}
&\\
10
&
    \frac{n(n + 2)}{n(2n+2)}
&\\
11
&
    \frac{n + 2}{2n + 2}  
&\\

12
&
    \frac{(n + 1) + 1}{2(n + 1)}  
&\\
13
&
    \therefore   \forall n \in \mathbb{Z}, ~
    n \geq 2
	~\rightarrow~
	\prod \limits_{i=1}^{n} \left (
		1 - \frac{1}{i^2}
	\right )
	~=~
	\frac{n+1}{2n}
    
&\\
\end{array}


\section{\exists n \in \mathbb{Z}, ~n > 0~\land~\prod \limits_{i=1}^{n} \left (\frac{1}{2i~+~1} \cdot \frac{1}{2i~+~2}\right )~\neq~\frac{1}{(2n~+~2)!}}

\begin{array}{l l l}
Row & Statement & Comment \\
1 &Let:n:=~1 ~\mid~ \frac{1}{2\cdot1+1}\cdot\frac{1}{2\cdot1+2}= \frac{1}{12}& \text{Base case and RHS} \\
2 &\frac{1}{(2\cdot1+2)!}=\frac{1}{24}& \text{LHS} \\
3 &\frac{1}{24} \neq \frac{1}{12}&\text{inductive hyp.} \\

\end{array}

\section
{
\forall m,n \in \mathbb{Z}, ~m,n \geq 1 ~\land~ m \equiv 1(\mod 2)~\rightarrow~m \mid \sum \limits_{i=0}^{m-1} (n ~+~ i)
}

\begin{array}{l l l}
Row & Statement & Comment \\
1 &Let: \exists r \in \mathBB{Z} | m = 2(r+1)&\text{by~definition~of~odd~int} \\
2 &\sum \limits_{i=0}^{m-1} (n ~+~ i) = \sum \limits_{i=0}^{(2r +1)-1} (n ~+~ i)& \text{substitute} \\
3 & \sum \limits_{i=0}^{(2r +1)-1} (n ~+~ i) ~=~ \sum \limits_{i=0}^{2r} (n ~+~ i)&\text{simplify} \\
4 &\sum \limits_{i=0}^{2r} (n ~+~ i)=\sum \limits_{i=0}^{2r} (n) + \sum \limits_{i=0}^{2r} (i)  = (2r + 1)n + \sum \limits_{i=0}^{2r} (i)&\text{substitution}\\
5 &(2r + 1)r +   \sum \limits_{i=0}^{2r}=  (2r + 1)n + \frac{2r(2r+1)}{2} &\text{}\\
6 &(2r + 1)n + \frac{2r(2r+1)}{2} =  (2r + 1)n + q(2r+1)&\text{}\\
7 &(2r + 1)n + r(2r+1) = (2r + 1)(n +r) = m(n+r)&\\
8 &(n + r) \in \mathbb{Z}&\\
9 &m | m(n+r) &\text{by~definition~of~divisibility~}\\
10 &\therefore \forall m,n \in \mathbb{Z}, ~m,n \geq 1 ~\land~ m \equiv 1(\mod 2)~\rightarrow~m \mid \sum \limits_{i=0}^{m-1} (n ~+~ i)&\\
\end{array}

\section
{\forall n \in \mathbb{Z},~ n \geq 0 ~\rightarrow~ 6 \mid 7^n ~-~ 1}

\begin{array}{l l l}
Row & Statement & Comment \\
1 &Let~n:=, 0~such~that~6 \mid 7^n ~-~ 1&\text{base case} \\
2 &7 ^ 0 = 1 -1 = 0& \text{} \\
3 &6 \cdot 0 = 0&\text{} \\
4 &Let: \exists p \in \mathbb{Z},such~that.~~6|7^{n + 1} -1 \rightarrow 7^{n+1} -1 = 6p&\text{inductive~hypothesis}\\
5 &7^{n+1} - 1 =7 \cdot 7^n -1&\text{}\\
6 &7^n -1 = (6 + 1) \cdot 7^n -1&\text{}\\
7 &(6 + 1) \cdot 7^n -1 = (6  \cdot 7^n + (7^n -1) = 6 \cdot 7^n + 6p&\text{by~hypothesis}\\
8 &6 \cdot 7^n + 6p = 6(7^n+p)&\\
9 &7^n + p \in \mathbb{Z}, such~that.~~6|7^{n + 1} -1&\text{by~closure}\\
10&\therefore\forall n \in \mathbb{Z},~ n \geq 0 ~\rightarrow~ 6 \mid 7^n ~-~ 1&\\
\end{array}

\section
{\forall n \in \mathbb{Z},~ n \geq 0 ~\rightarrow~ 5 \mid7^n~-~ 2^n
}

\begin{array}{l l l}
Row & Statement & Comment \\
1 
&
Let~n:=0, ~such~that, 5 \mid 7^n - 2^n
&
\text{base case} \\

2 
&
7^0 - 2 ^0 = 1-1 = 0
& 
\text{} \\

3 
&
5 \cdot 0 = 0
&\text{} \\

4
&
     5 \mid 7^n ~-~ 2^n  ~\rightarrow~ 5\mid 7^{n+1} ~-~ 2^{n+1}  
&\text{inductive~hypothesis}\\

5
&
    Let:~ \exists p \in \mathbb{Z}~such~that~5\mid 7^n ~-~ 2^n  ~\rightarrow~ 7^n ~-~ 2^n = 5p
&\text{substitution~def.~of~divisibility}\\

6
&
    7^{n+1} - 2^{n+1} = 7\cdot 7^n - 2 \cdot 2^n = (5+2)\cdot 7^n - 2\cdot 2^n
&\text{}\\

7
&
    5\cdot 7^n - 2\cdot 2^n = 5\cdot7^n+2\cdot7^n-2\cdot2^n = 5\cdot 7^n + 2(7^n-2^n)
&\text{algebra}\\

8
&
    5\cdot 7^n + 2(7^n-2^n) ~=~ 5\cdot 7^n + 2\cdot 5p 
&\text{by~hypothesis}\\

9
&
 5\cdot 7^n + 2\cdot 5p ~=~5(7^n + 2p), \in \mathbb{Z}
&\text{}\\

10
&
5 | 5(7^n + 2p)
&\\

11
&
\therefore \forall n \in \mathbb{Z},~ n \geq 0 ~\rightarrow~ 5 \mid 7^n ~-~ 2^n
&\\
\end{array}

8
\section
{
\forall n \in \mathbb{Z},~ n \geq 2 ~\rightarrow~ \sqrt{n} ~<~ \sum \limits_{i=1}^{n} \left ( \frac{1}{\sqrt{i}} \right )
}

\begin{array}{l l l}
Row & Statement & Comment \\
1 
&
Let~n:=2, ~such~that,  \sqrt{n} ~<~ \sum \limits_{i=1}^{n} \left ( \frac{1}{\sqrt{i}} \right )
&
\text{base case} \\

2 
&
\sqrt{2} < \frac{1}{\sqrt{1}} + \frac{1}{\sqrt{2}} ~i.f.f~2<\swrt{2} + 1 
& 
\text{} \\

3 
&
1<\sqrt{2} 
&\text{algebra} \\

4
&
\sqrt{n} ~<~ \sum \limits_{i=1}^{n} \left ( \frac{1}{\sqrt{i}} \right )\rightarrow \sqrt{n+1} ~<~ \sum \limits_{i=1}^{n+1} \left ( \frac{1}{\sqrt{n+1}} \right )
       
&\text{inductive~hypothesis}\\

5
&
    \sqrt{n} < \sqrt{n+1} = n < \sqrt{n}\cdot\sqrt{n+1} = n+1 < \sqrt{n}\cdot\sqrt{n+1}+1
&\text{by~algebra~LHS}\\

6
&
    n+1 < \sqrt{n}\cdot\sqrt{n+1}+1 ~=~ \sqrt{n+1} < \sqrt{n}+\frac{1}{\sqrt{n+1}}
&\text{}\\

7
&
    \sqrt{n+1} <  \sum \limits_{i=1}^{n+1} \left ( \frac{1}{\sqrt{n+1}} \right)
&\text{by~subtitution}\\

8
&
    \therefore \forall n \in \mathbb{Z},~ n \geq 2 ~\rightarrow~ \sqrt{n} ~<~ \sum \limits_{i=1}^{n} \left ( \frac{1}{\sqrt{i}} \right )
&\text{QED}\\
\end{array}
9
\section{\begin{array}{l}
a_1 ~=~ 1 \\
a_2 ~=~ 3 \\
\forall k \geq 3,~a_k ~=~ a_{k-1} ~+~ a_{k-2}\\
\forall n \in \mathbb{Z}, n \geq 1 ~\rightarrow~ a_n \leq \left ( \frac{7}{4} \right )^n\\
\end{array}}

\begin{array}{l l l}
Row & Statement & Comment \\
1 &Since: a_1=1, a_2 = 3, \frac{7}{4} > 1 \land (\frac{7}{4})^2 = \frac{49}{16} = 3\frac{1}{16} > 3& \text{base case} \\
2 & P(n)~=~\forall n \in \mathbb{Z}, n \geq 1 ~\rightarrow~ a_n \leq \left ( \frac{7}{4} \right )^n& \text{} \\
3 & \forall k \geq 2 [\forall i, 1 \leq i \leq k, P(i)] \rightarrow P(k+1) & \text{} \\
4 &\forall i,~a_i \leq (\frac{7}{4})^i \rightarrow \forall k,~a_{k+1} \leq \frac{7}{4}^{k+1} & \text{inductive~hyp.} \\
5 &Since: k \geq 2, a_{k+1} = a_k + a_{k-1} & \text{by~definition} \\
6 &a_k + a_{k-1} \leq (\frac{7}{4})^k+(\frac{7}{4})^{k-1}& \text{substitute~by~hyp} \\
7 &a_k + a_{k-1} \leq (\frac{7}{4})^{k-1}(\frac{7}{4}+1)& \text{factor} \\\\
8 &a_k + a_{k-1} \leq (\frac{7}{4})^{k-1}(\frac{11}{4})\cdot \frac{4}{4} = (\frac{7}{4})^{k-1}(\frac{44}{16})& \text{} \\
9 &a_k + a_{k-1} \leq  (\frac{7}{4})^{k-1}(\frac{49}{16})  =a_k + a_{k-1} \leq  (\frac{7}{4})^{k-1}(\frac{7}{4})^2&\\
10 &a_k + a_{k-1} \leq  (\frac{7}{4})^{k-1}(\frac{7}{4})^2=a_k + a_{k-1} \leq  (\frac{7}{4})^{k+1}&\text{} \\
11&& \text{}
12&\therefore \forall n \in \mathbb{Z}, n \geq 1 ~\rightarrow~ a_n \leq \left ( \frac{7}{4} \right )^n&\\
\end{array}
10
\section{
i,m,largest,A[i] \in \mathbb{Z} \\
while(i \neq m) ~\ {\\
~1. ~i ~:=~ i~+~1\\
~2. ~if~A[i] > largest ~then~ largest:=A[i]\\
\ }\\
}

\begin{array}{ll}
Basis: & \\
1.~ I(i) ~=~Largest = max~A[1]~\land~i=1 & \text{by precondition} \\
2.~\therefore I(0) ~\equiv~ true& \text{By pre-condition} \\
Induction: & \\
1.~ k \in \mathbb{Z}, k \geq 0 ~\mid~ G \land I(k) ~\equiv~ true & \text{k is a generic particular} \\
2.~i \neq m ~\land~ i = k+1 ~\land~ i=l & \text{At start of loop, before (1)} \\
3.~i_{old}=k+1 ~\land~ i_{new}-i_{old}+1 = k+2 & \text{Effect of (1)} \\
4.~A[i_{new}]>Largest_{old}, \rightarrow ~Largest_{new}=A[i_{new}]& \text{Effect of (2)(1)} \\
5.~~A[i_{new}]  \leq~Largest_{old} \rightarrow  Largest_{new}=Largest_{old}& \text{effect(2)(2)} \\
6.~Largest_{new}= A[1],A[2],A[3],...,A[k+1],A[k+2])& \text {for both cases}\\
70.~\therefore I(k+1) \equiv true & \text{Inductive QED}
Falsity Of Guard: & \\
1. i = m&\\
Post~Condition:\\
1.~Suppose: N is~min~iteration~where~G\equiv false \land I(N) \equiv true = m-1&\\
2.~Since:~ G \equiv false, i = m &\\
3.~Since:~ I(n) \equiv true&\\
4.~Largest~=~Max~of~ A[1],A[2],A[3],...,A[m-1],A[m]&\\
\end{array}
11
\section{
\forall n \in \mathbb{Z}, ~n \geq 1 ~\rightarrow~ S_n ~=~ \frac{-S_{(n-1)}}{n}
}

\begin{array}{lll}
1.&\forall n \in \mathbb{Z}, s_0,s_1,s_2,s_3....~\mid~ n \geq 0  \rightarrow s_n=\frac{(-1)^n}{n!}& \\
2.&\exists x \in \mathbb{Z}~\mid~s_x=\frac{(-1)^{x}}{x!}~\land~s_{x-1}=\frac{(-1)^{x-1}}{(x-1)!} &  \\
3.&\frac{-s_{x-1}}{x}=\frac{-\frac{-1^{x-1}}{(x-1)!}}{(x)}&\\
4.&\frac{-\frac{-1^{x-1}}{(x-1)!}}{(x)} = \frac{-(-1)^{x-1}}{x(x-1)!}&\\
5.&\frac{-(-1)^{x-1}}{x(x-1)!} = \frac{(-1)^x}{x!}&\\
6.&s_k&\\
7.&\therefore &\\
\end{array}


\section{
\forall n \in \mathbb{Z}, ~  n \geq 1 ~\rightarrow~ F_n < 2^n
}

\begin{array}{lll}
1.&F_0=1,F_1=1& \\
2.&\forall F_x \in \mathbb{Z}~\mid~x \geq 2 \rightarrow F_x=F_{x-1}+F_{x-2} &  \\
3.&Let:~n:=1 ~\mid~F_1=1<2^1&\\
4.&\forall n \in \mathbb{Z}, ~  n \geq 1 ~\rightarrow~ F_n < F_{n+1} < 2^{n+1}&\\
5.&F_{n+1}=F-n+F_{n-1}&\\
6.&< 2^k+2^{k-1}&\\
7.&< 2^k+2^k&\\
8.&< 2\cdot2^k&\\
9.&< 2^{k+1}=F_{n+1}<2^{n+1}&\\
\end{array}

\end{document}

