
\documentclass[10pt,letterpaper, cm]{hmcpset}
\usepackage[top=0.5in,bottom=1in,right=1in,left=0.75in]{geometry}
\usepackage{multicol,graphicx, enumerate, cancel, amsmath, amsthm}
\usepackage{tikz}
\usepackage{tikz,fullpage}

\usetikzlibrary{arrows,petri,topaths,shapes, backgrounds}
				
\usepackage{tkz-berge}
\usepackage{tkz-graph}

\name{Jonathon Sonesen}
\class{CS 251}
\duedate{\today}
\assignment{Final Exam (Of Doom)}
\extraline{PCC G number: G03584121}
\setlength{\columnsep}{0.5cm}
\setlength{\columnseprule}{00pt}


\def\firstcircle{(0,0) circle (1.5cm)}
\def\secondcircle{(45:2cm) circle (1.5cm)}
\def\thirdcircle{(0:2cm) circle (1.5cm)}

\begin{document}
\begin{problem}[1]
State precisley (but concisely) in your own words what it mean for an argument to be valid.
\end{problem}\\
\\An argument is considered valid if it's premise and conclusion are logically sound. \\

\begin{problem}[2]
  Prove (or disprove).\\
  \begin{center}
    \begin{equation*}
      \forall n \in \mathbb{Z}, n \geq 3 \rightarrow \sum\limits_{i=3}^{n}i(i-1) 
      =  \frac{(n-2)(n^2+2n+3)}{2}
    \end{equation*}
  \end{center}
\end{problem}\\

\\
Proof:
\begin{center}
  \begin{equation*}
    \exists n \in \mathbb{Z}, n \geq 3 \land  \sum\limits_{i=3}^{n}i(i-1) 
      \neq \frac{(n-2)(n^2+2n+3)}{2}
  \end{equation*}
\end{center}
\begin{center}
\begin{array}{lll}
  Row & Statement & Comment\\
  1.& Let~n:= 3&\text{counter~example}\\
  2.& \sum\limits_{i=3}^{3}i(i-1) = 6 &\\
  3.&\frac{(3-2)(3^2+2 \cdot 3+3)}{2} = 9 & \\
  4.&6 \neq 9 &\\
  5.& \therefore \exists n \in \mathbb{Z}, n \geq 3 \land  \sum\limits_{i=3}^{n}i(i-1) 
  \neq \frac{(n-2)(n^2+2n+3)}{2} & \text{QED}\\
\end{array}
\end{center}

\begin{problem}[3]
  Prove (or disprove)\\
  \begin{equation*}
    \forall~sets~A,B,C~\in \matbb{U} \mid A \cup (B \cap C) \subset (A \cup B) \cap C
  \end{equation*}
\end{problem}\\
\\
\\
Proof:\\
\begin{equation*}
  \exists~sets~A,B,C~\in \mathbb{U}~\mid A \cup (B \cap C) \not\subset (A \cup B) \cap C
\end{equation*}

\begin{array}[]{lll}
Row & Statement & Comment\\
1.  & Let~A=\left\{ 1,2,4,5 \right\}&\text{counter~example}\\
2.  & Let~B=\left\{ 2,3,5,6 \right\}&\\
3.  & Let~C=\left\{ 4,5,6,7 \right\}&\\
4.  & Say~\exists~some~set~X~\mid~X=A \cup (B\cap C)  = \left\{ 1,2,5,4,6\right\}&\\
6.  & Say~\exists~some~set~Y~\mid~Y=(A\cup B) \cap C = \left\{ 2,5,4,6 \right\}&\\
7.  & 1 \not\in Y &\\
8.  & X \not\subset Y &\\
9.  & \therefore 
\exists~sets~A,B,C~\in \mathbb{U}~\mid A \cup (B \cap C) \not\subset (A \cup B) \cap C&
\text{QED}\\
\end{array}

\newpage
\begin{problem}[4]
  Let $F:\mathbb{R}~x~\mathbb{R} \rightarrow \mathbb{R}~x~\mathbb{R}$ 
  be a relation defined as\\
  \begin{equation*}
    F(x,y) = (3y-1,1-x)
  \end{equation*}
  \begin{center}
    \begin{enumerate}[(a)]
      \item Prove or disprove that F is a function.\\
      \item Prove (or disprove) $ F^{-1}$ is a function\\
    \end{enumerate}
  \end{center}
\end{problem}

\begin{enumerate}[(a)]
  
  \item 
    
        \begin{array}[]{lll}
        Row & Statement & Comment\\
        1.  & \forall~F(x,y)~\exists~z \in \mathbb{R} \mid z = F(x,y) \rightarrow 
              z \in \mathbb{R}& \text{By~closure~on~addition,~and,~subtraction}\\
        2.  &\therefore F(x,y)~is~a~function&QED\\ 
        
        \end{array}
  \item 
    
        \begin{array}[]{lll}
        Row & Statement & Comment\\
        1.  & F(x,y) = (3y-1,1-x) \rightarrow F^{-1}=(1-y,\frac{x+1}{3}
            & \text{By~definition~of~inverse~function}\\
        2.  &\ \forall~F(x,y)^{-1}~\exists~z \in \mathbb{R} \mid z = F^{-1}(x,y) 
            \rightarrow z \in \mathbb{R}& \text{By~closure~on~~subtraction,~division}\\ 
            3.  & \therefore~F^{-1}~is~a~function&QED\\
        \end{array}
\end{enumerate}

\begin{problem}[5]
Let: \\
\begin{center}
  \begin{equation*}
    A= \left\{ x \in \mathbb{Z} \mid \exists~k \in \mathbb{Z} \land x = k^2 \right\}
  \end{equation*}
\end{center}
What is $\mid A \mid $? Explain your reasoning and justify your answer.
\end{problem}\\
\\
$\mid A \mid $ is countably infinite. This is because for every integer in  the set of all
integers that exists there is a square that also exists. In other words,
can be put into a one to one correspondence with the set of all integer squares\\

\begin{problem}[6]
Define A to be the set of unique digits on your PCC G-Number, and let R :$A\rightarrow A$ be\\
deifined as\\
\begin{center}
  \begin{equation*}
    xRy \leftrightarrow 2\mid(x-y)
  \end{equation*}
\end{center}\\
List the equivalence classes of R or prove no such classes exist.
\end{problem}\\
\\
\begin{center}
  $ A =  \left\{ 0,1,2,3,4,5,8 \right\}$\\
\end{center}
 The equivalence classes are:\\
 \begin{center}
   \left\{ 0,2,4,8 \right\},
   \left\{ 1,3,5 \right\},
   \left\{ \emptyset \right\}&
 \end{center}
\newpage
\begin{problem}[7]
  Let S be the set of all strings of 0's and 1's of length 3. Define $ R:S \rightarrow S$ as\\ 
  \begin{center}
    \begin{align*}
    ~the~two~left~most~characters& \\
      sRt \leftrightarrow~
      of~s~are~the~same~as~the~two\\
      left~most~characters~of~t 
    \end{align*}
  \end{center}
  List the equivalence classes of R or prove no such classes exist.
\end{problem}\\
\\
\begin{center}
  $ S=\left\{ 000, 100, 010, 001, 111, 011, 101, 110 \right\}$\\
\end{center}
\\
 The equivalence classes are:
\begin{align*}
   \left\{ 000, 001 \right\},
   \left\{ 001, 010 \right\},
   \left\{ 101, 100 \right\},
 \left\{ 110, 111 \right\},
 \left\{ \emptyset \right\} 
\end{align*}
\begin{problem}[8]
  Let $x$ be you PCC G number without the leading G. What is the inverse modulo of
  x modulo 8831?
\end{problem}\\
\\
The inverse modulo does not exist. My G number and 8331 are not coprime.\\
\\
\begin{problem}[9]
  An RSA Cipher has the public key pq=65 and e=7. What is the encrypted value of the
  last 3 digits of your PCC G number?
\end{problem}\\
\\
The encrypted values of 1, 2, 1 are:
1, 63, 1
\\

\begin{problem}[10]
  Three quizzes are given to a class of 30 students, and all student submitted all quizzes.
  Given:\\
  \begin{enumerate}[-]
    \item 15 students scored 12 or more on quiz 1
    \item 12 students scored 12 or more on quiz 2
    \item 18 students scored 12 or more on quiz 3
    \item 7 students scored 12 or more on quizzes 1 and 2
    \item 11 students scored 12 or more on quizzes 1 and 3
    \item 8 students scored 12 or more on quizzes 2 and 3
    \item 4 students scored 12 or more on quizzes 1, 2, and ,3
  \end{enumerate}
  How many students scored 12 or more on quizzes 1 and 2 but not 3?
\end{problem}\\

Three students scored 12 or more on quzzes 1 and 2 but not 3.


\begin{problem}[11]
  An urn contains four balls, each numbered with one of the last 4 digits
  of your PCC G-Number. If a person selects two balls at random (equal
  probability), what is the expected value of the product of the numbers on
  the balls?
\end{problem}\\

The expected value of the product of the numbers on the balls is about 5.1667.

\begin{problem}[12]
   If a graph has nodes of degrees 1, 1, 2, 3, and 3, how many edges does it
   have? Explain your reasoning and justify your answer, although a formal
   proof is not needed.
\end{problem}\\
\begin{center}

  $1 + 1 +2 + 3 + 3 + 3 = 10$\\
  By thm. 10.1.1 the sum of the degrees is equal to the twice the number of edges.\\
  $10=2\cdot k$ where k is the number of edges so   
  $k=5$\\
\end{center}

\begin{problem}[13]
   Specify (drawing or adjacency matrix) a full binary tree with 16 nodes, of
   which 6 are internal nodes, or prove no such graph exists.   
\end{problem}\\

\begin{center}
Suppose not:\\
\begin{array}{lll}
  Row & Statement & Comment\\
  1.& A~full~binary~tree~with~k~nodes~has~2k+1~internal~nodes&\text{thm~10.6.1~Susanna~Epp}\\
  2. &k=6& \\
  3. &2\cdot 6 +1 = 13&\\
  4. &A~full~binary~tree~with~6~internal~nodes~would~have~a~total~of~13~nodes&\\
  5. &\therefore~no~such~graph~exists&\text{QED}\\
\end{array}
\end{center}

\begin{problem}[14]
  Must a graph with 68 nodes and 72 edges have a circuit? Explain your
  reasoning and justify your answer, although a formal proof is not needed.  
\end{problem}\\
\\
,No, it could be a tree.
\\
\begin{problem}[15]
  Given the adjacency matrix:\\
  \begin{center}
      AM = \begin{pmatrix}

        0 & 1  & 0 & 1\\
        1 & 0 & 2 & 0\\
        0 & 2 & 0 & 0\\
        1 & 0 & 0 & 0\\
      \end{pmatrix}
  \end{center}
Assume the ows correspond to nodes 1, 2,3, and 4 of the graph. How many walks of 
length 15 are there from node 3 to node 1?
\end{problem}\\
\\
Zero Walks.
\end{document}
